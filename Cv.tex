\documentclass[a4paper,10pt]{article}

% Engine: xelatex
\usepackage[a4paper,margin=1.2cm]{geometry}
\usepackage{fontspec}
\setmainfont{Arial}

\usepackage{paracol}        % two columns
\usepackage{graphicx}
\usepackage[dvipsnames]{xcolor}
\usepackage{enumitem}
\usepackage{tikz}
\usepackage{fontawesome5}
\usepackage[hidelinks]{hyperref}
\usepackage{titlesec}
\usepackage[most]{tcolorbox}
\usepackage{tabularx}
\usepackage{pgffor}
\usepackage{setspace}

% Colors (close to the sample)
\definecolor{leftbg}{HTML}{2F2F37}
\definecolor{accent}{HTML}{4B28A4}
\definecolor{muted}{HTML}{6E6E78}
\definecolor{textcol}{HTML}{111111}

% Section style
\titleformat{\section}{\Large\bfseries\color{textcol}}{}{0pt}{}[\color{accent}\titlerule]
\titlespacing{\section}{0pt}{4pt}{4pt}
\setlength{\parskip}{0.3em}
\renewcommand{\baselinestretch}{1.02}

% Helpers
\newcommand{\dotbar}[1]{%
  % 1..5 filled dots
  \foreach \i in {1,...,5}{%
    \ifnum\i<=#1\textcolor{accent}{\faCircle}\else\textcolor{muted}{\faRegCircle}\fi
    \hspace{0.15em}%
  }%
}

\newcommand{\sideheader}[1]{\vspace{0.8em}\textcolor{white}{\large\bfseries #1}\par\vspace{0.3em}}
\newcommand{\sideitem}[2]{\textcolor{white}{#1}\par\textcolor{muted}{\small #2}\par\vspace{0.35em}}
\newcommand{\sidelink}[2]{\textcolor{white}{#1}\par\textcolor{muted}{\small #2}\par\vspace{0.35em}}

\pagestyle{empty}

\begin{document}
\color{textcol}

% Two-column layout
\setlength{\columnsep}{1.1cm}
\begin{paracol}{2}

% ---------------- LEFT BAR ----------------
\begin{tcolorbox}[
    colback=leftbg, colframe=leftbg, boxrule=0pt, sharp corners,
    left=8mm, right=8mm, top=8mm, bottom=8mm,
    enlarge left by=-1.2cm, enlarge right by=-\columnsep,
    height=\textheight
]
% Name
{\color{white}\fontsize{24}{26}\selectfont ROMAIN DESPOULLAINS}\par
\vspace{0.6em}
{\color{muted}\small \faMapMarker*\, Paris}\par

% Photo (replace filename if needed)
\vspace{0.8em}
\begin{center}
\begin{tikzpicture}
  \clip (0,0) circle (1.65cm);
  \node at (0,0) {\includegraphics[width=3.3cm]{AI Profile Picture.png}};
\end{tikzpicture}
\end{center}
\vspace{0.6em}

% Contact
\sideheader{Données personnelles}
\textcolor{white}{\small \faEnvelope\ \href{mailto:romaindespoul@gmail.com}{romaindespoul@gmail.com}}\par
\textcolor{white}{\small \faMobile\ +33 7 88 69 26 08}\par
\textcolor{white}{\small \faGithub\ \href{https://github.com/wolf75222}{github.com/wolf75222}}\par
\textcolor{white}{\small \faLinkedin\ \href{https://linkedin.com/in/romain-despoullains}{linkedin.com/in/romain-despoullains}}\par

% Languages
\sideheader{Langues}
\textcolor{white}{Français}\hfill \dotbar{5}\par
\textcolor{white}{Anglais}\hfill \dotbar{4}\par

% Soft skills / traits
\sideheader{Compétences}
\textcolor{white}{Compétences en leadership}\par
\textcolor{white}{Gestion}\par
\textcolor{white}{Axé sur les processus}\par
\textcolor{white}{Souci du détail}\par

% Interests
\sideheader{Intérêts}
\textcolor{white}{HPC, IA, Robotique, Jeux vidéo}\par

\end{tcolorbox}

% ---------------- RIGHT COLUMN ----------------
\switchcolumn

% Title / Summary
{\Large\bfseries Ingénierie informatique — HPC \& IA}\par
\vspace{0.3em}
\onehalfspacing
Étudiant en ingénierie informatique, passionné par les nouvelles technologies, avec esprit d’innovation, autonomie et sens de l’initiative dans la résolution de problèmes complexes, à l’aise en travail d’équipe et communication technique. \\
Actuellement en Master Calcul Haute Performance et Simulation, en recherche d’alternance/stage en HPC et IA.
\setstretch{1.12}
\vspace{0.6em}

% Technical skills
\section{Compétences techniques}
\textbf{Langages et paradigmes:} C/C++, Rust, C\#, Java, Python, TypeScript/JavaScript, PHP, SQL/PLSQL, CUDA, MPI, OpenMP, NASM, LaTeX.\\
\textbf{Technologies et outils:} React, Docker, Kubernetes, Slurm, Git, CI/CD (GitLab CI, Jenkins), VMWare/VirtualBox/Proxmox/QEMU/PXE, Ansible, Zabbix, Grafana, Prometheus, MySQL/SQLite, Spring Boot, Django/Flask/Laravel/Android, Postman, LangChain/LangFlow/Autogen/ONNX, OpenGL, Ray Tracing.

% Education
\section{Éducation}
\begin{tabularx}{\linewidth}{@{}X r@{}}
\textbf{Université de Reims Champagne-Ardenne} — CMI: informatique et simulation numérique & 2022–2027 \\
\multicolumn{2}{@{}X@{}}{\emph{Mention Très Bien}. Formation adossée à la licence et au master d’informatique, axée recherche, méthodologie scientifique, anglais scientifique, communication, gestion de projet, mathématiques appliquées, physique et programmation avancée.} \\
\end{tabularx}
\vspace{0.4em}
\begin{tabularx}{\linewidth}{@{}X r@{}}
\textbf{Université de Reims Champagne-Ardenne} — Licence 3 Informatique & 2022–2025 \\
\multicolumn{2}{@{}X@{}}{\emph{Mention Très Bien}. Cours: sécurité, virtualisation et cloud, admin système, programmation système et multi-thread, BD avancées, répartie, HPC (NVIDIA CUDA Fundamentals), IA et données, crypto, complexité et calculabilité, langages, graphes, synthèse d’images.} \\
\end{tabularx}

% Academic experience
\section{Expérience académique}
\textbf{Travail d’étude et de recherche — Laboratoire LAB-I*} \hfill 2024–2025 \\
\begin{itemize}[leftmargin=1.2em,itemsep=2pt,topsep=2pt]
  \item Solution répartie de résolution du problème des n-dames dans le cloud, avec WebAssembly et file de messages AMQP.
  \item Génération de scénarios (CARLA, AirSim) pour tests de véhicules autonomes via IA générative.
\end{itemize}

\textbf{Travail d’étude et de recherche — Laboratoire LICIIS} \hfill 2023–2024 \\
\begin{itemize}[leftmargin=1.2em,itemsep=2pt,topsep=2pt]
  \item Étude sur la détection de données générées par IA.
\end{itemize}

% Professional
\section{Expérience professionnelle}
\textbf{Latitude (Aérospatiale) — Stage Ingénieur HPC \& IA} \hfill Sept. 2024 – Août 2025 \\
\begin{itemize}[leftmargin=1.2em,itemsep=2pt,topsep=2pt]
  \item Développement d’un connecteur (API C++/Java) et d’applications desktop pour l’écosystème PLM Siemens (Teamcenter, NX, StarCCM++).
  \item Optimisation d’un algorithme génétique continu pour trajectoires de fusées (+20\%), benchmarks (CMA-ES, DE, GA) et déploiement sur supercalculateur via Slurm.
  \item Conception d’une application de visualisation des trajectoires de fusées (Agile/Scrum).
  \item Mise en place de solutions IA (RAG, Modern Data Stack) pour l’intégration et l’exploitation de données métier.
\end{itemize}

\textbf{LAB-I* (Transports Intelligents) — Stage Data Scientist} \hfill Avril 2024 – Juil. 2024 \\
\begin{itemize}[leftmargin=1.2em,itemsep=2pt,topsep=2pt]
  \item Applications web (Flask, Python) pour analyse statistique avancée et détection d’anomalies sur systèmes de transport intelligents.
  \item Automatisation de l’analyse des messages inter‑véhicules, étude statistique des profils de réception et analyse de tendances.
  \item Participation au projet européen InDiD avec acteurs publics et industriels (Commission Européenne, MTE, Valeo, APRR), conformité ETSI (DENM, CAM, C‑ITS, GeoNetworking).
\end{itemize}

% Engagement
\section{Engagement \& bénévolat}
\textbf{Association MIRAGE — Vice-président, ancien Président} \hfill 2023–2025 \\
\begin{itemize}[leftmargin=1.2em,itemsep=2pt,topsep=2pt]
  \item Direction stratégique et opérationnelle de l’association, coordination des équipes, événements et projets, gestion financière et communication.
\end{itemize}

\textbf{Université de Reims \& Indépendant — Enseignant vacataire étudiant, Professeur particulier} \hfill 2021–2025 \\
\begin{itemize}[leftmargin=1.2em,itemsep=2pt,topsep=2pt]
  \item Tutorat L1 Info/Maths, cours particuliers (maths, physique, NSI), rédaction de supports en \LaTeX.
\end{itemize}

% References
\vspace{0.4em}
\textit{Références disponibles sur demande.}

\end{paracol}
\end{document}
